%JULIA HEADER
\documentclass{article}
\usepackage{amsmath,amsthm}
\usepackage{amssymb,latexsym}
\usepackage{epsfig}
\usepackage{hyperref}
\usepackage{float}
\usepackage{fullpage}
\usepackage{enumerate}
\usepackage{paralist}
\usepackage{times}


\newtheorem{theorem}{Theorem}
\newtheorem{corollary}[theorem]{Corollary}
\newtheorem{question}[theorem]{Question}
\newtheorem{lemma}[theorem]{Lemma}
\newtheorem{observation}[theorem]{Observation}
\newtheorem{proposition}{Proposition}
\newtheorem{definition}[theorem]{Definition}
\newtheorem{claim}[theorem]{Claim}
\newtheorem{fact}[theorem]{Fact}
\newtheorem{assumption}[theorem]{Assumption}
\newtheorem{example}{Example}
\newtheorem{conjecture}[theorem]{Conjecture}
\newtheorem{alg}[theorem]{Algorithm}

\newcommand{\myparagraph}[1]{\paragraph{#1.}}

\newcommand{\eps}{\varepsilon}
\newcommand{\epssdp}{\varepsilon_{\rm sdp}}

\newcommand{\C}{C}
\newcommand{\Tr}{Tr} %CHECK
\newcommand{\Id}{Id} %CHECK
\newcommand{\Exs}[2]{E_{#1}[#2]} %CHECK

\newcommand{\trace}{{\rm Tr}}

\newcommand{\norm}[1]{\left\|\,#1\,\right\|}       % norm
\newcommand{\onorm}[1]{\norm{#1}_{\mathrm{1}}}      % Euclidean norm for vectors
\newcommand{\enorm}[1]{\norm{#1}_{\mathrm{2}}}      % Euclidean norm for vectors
\newcommand{\trnorm}[1]{\norm{#1}_{\mathrm {tr}}}  % trace norm
\newcommand{\fnorm}[1]{\norm{#1}_{\mathrm {F}}}    % frobenius norm
\newcommand{\snorm}[1]{\norm{#1}_{\mathrm {\infty}}}    % spectral norm

\newcommand{\set}[1]{{\left\{#1\right\}}}    % braces for set notation
\newcommand{\ve}[1]{\mathbf{#1}}
\newcommand{\abs}[1]{\left\lvert #1 \right\rvert}

\newcommand{\complex}{{\mathbb C}}
\newcommand{\reals}{{\mathbb R}}
\newcommand{\ints}{{\mathbb Z}}
\newcommand{\nats}{{\mathbb N}}

\newcommand{\proj}[1]{\mbox{$|#1\rangle \!\langle #1 |$}}
\newcommand{\enc}[1]{\left<#1\right>}

\newcommand{\spa}[1]{\mathcal{#1}}
\newcommand{\dens}{D(\spa{A}\otimes\spa{B})}
\newcommand{\unitaries}{U(\spa{A}\otimes\spa{B})}

\bibliographystyle{alpha}

\begin{document}

\title{CMSC 303 Introduction to Theory of Computing, VCU\\Spring 2017, Assignment 3\\Due: Thursday, February 23, 2017\\Matthew Bowers}
\date{}
\maketitle
\noindent Total marks: $66$ marks $+$ $4$ marks bonus $+$ $8$ bonus marks for LaTeX\\

\noindent Unless otherwise noted, the alphabet for all questions below is assumed to be $\Sigma=\set{0,1}$.
%\section{Questions}
\begin{enumerate}
    \item {[12 marks]} This question develops your ability to devise regular expressions, given an explicit definition of a language. For each of the following languages, prove they are regular by giving a regular expression which describes them. Justify your answers.
        \begin{enumerate}
            \item $L=\set{x\mid x \text{ begins with a $0$ and ends with a $1$}}$.\\
            \\
            $0\Sigma^*1$\\
            \item $L=\set{x\mid x \text{ contains at least four $0$'s}}$\\
            \\
            $\Sigma^*0\Sigma^*0\Sigma^*0\Sigma^*0\Sigma^*$\\
            \item $L=\set{1, 11, \epsilon}$.
            \item $L=\set{x\mid \text{the length of $x$ is at most $3$}}$.
            \item $L=\set{x\mid x \text{ doesn't contain the substring $110$}}$.
            \item $L=\set{x \mid \abs{x}>0 \text{, i.e. $x$ is non-empty}}$.
        \end{enumerate}
    \item {[20 marks]} This question tests your understanding of how to translate a regular expression into a finite automaton. Using the construction of Lemma 1.55, construct NFAs recognizing the languages described by the following regular expressions.
        \begin{enumerate}
            \item {[5 marks]} $R=\emptyset^*$.
            \item {[15 marks]} $R=(0\cup 1)^*111(0\cup 1)^*$.
        \end{enumerate}
    \item {[15 marks]} This question tests your understanding of how to translate a finite automaton into a regular expression. Consider DFA $M=(Q,\Sigma,\delta,q,F)$ such that $Q=\set{q_1,q_2,q_3}$, $q=q_1$, $F=\set{q_1,q_3}$, and $\delta$ is given by:
                \[
\begin{tabular}{|c|cc|}
  \hline
  $\delta$ & 0 & 1 \\
  \hline
  $q_1$ & $q_2$ & $q_2$\\
  $q_2$ & $q_2$ & $q_3$\\
  $q_3$ & $q_1$ & $q_2$\\
  \hline
\end{tabular}
        \]
            Draw the state diagram for $M$, and then apply the construction of Lemma 1.60 to obtain a regular expression describing $L(M)$.
    \item {[15 marks]} This question allows you to practice proving a language is non-regular via the Pumping Lemma. Using the Pumping Lemma (Theorem 1.70), give formal proofs that the following languages are \emph{not} regular:
        \begin{enumerate}
            \item $L=\set{www\mid w\in\set{0,1}^*}$.
            \item $L=\set{1^n0^m1^n\mid m\text{, }n\geq 0}$.
            \item $L=\set{x\mid x\in\set{0,1}^*\text{ is not a palindrome}}$. Recall a palindrome is a string that looks the same forwards and backwards. Examples of palindromes are ``madam'' and ``racecar''.
        \end{enumerate}
    \item  {[$4$ marks $+$ $4$ marks bonus]} This question reveals important subtleties of the Pumping Lemma. Let $B=\set{0^kx0^k \mid k\geq 1\text{ and } x\in\Sigma^*}$.
                \begin{enumerate}
                \item{[4 marks]} Consider the following argument, which claims to prove that $B$ is not regular.

                    Assume $B_2$ is regular, and let $p$ be the pumping length. Consider string $s=0^p10^p\in B_2$, and decompose it as $s=xyz$ with $x=\epsilon$, $y=0^p$, $z=10^p$. Then, pumping $s$ down by setting $i=0$ yields string $s'=xy^iz=xy^0z=10^p\not \in B_2$. Hence, by the Pumping Lemma, we have a contradiction. We conclude that $B_2$ is not regular.

                    The question is: What is wrong with this proof?
                \item {[4 marks, bonus]} Prove that, in fact, $B$ is regular.
            \end{enumerate}
\end{enumerate}
\end{document}
